% Chapter 4

\chapter{其它格式}
\section{代码}
\subsection{原始代码}
朴实的代码块:

使用 \env{verbatim} 环境可以得到如下原样的输出。
\begin{verbatim}
print("Hello world!")
\end{verbatim}

使用 \pkg{listings} 包提供的 \env{lstlisting} 环境可以对代码进行进一步的格式化。
\begin{lstlisting}[language=Python,frame=single]
import numpy as np

a = np.zeros((2,2))
print(a)
\end{lstlisting}

\subsection{代码高亮}
\pkg{minted} 包所提供的 \env{minted} 环境还可以对代码进行高亮,请参考 \href{https://www.overleaf.com/learn/latex/Code_Highlighting_with_minted}{Code Highlighting with minted} 进行调试。在使用 \env{minted} 环境前,请先在 \cls{whuthesis.cls} 文件中启用 \pkg{minted} 包。

\begin{notice}
  使用 \pkg{minted} 包时,需要系统拥有 \app{Python} 环境,并安装 \app{Pygments} 包,可以通过 \verb|$ pip install Pygments| 来进行安装。且需要在编译时加上 \verb|-shell-escape| 参数,否则会报错。
\end{notice}

% \usemintedstyle{vs}
% \begin{minted}[linenos,baselinestretch=1.0,frame=lines]{cpp}
% #include <iostream>
% using namespace std;

% int main() 
% {
%     cout << "Hello, World!";
%     return 0;
% }
% \end{minted}

\subsection{算法描述/伪代码}
参考 \href{https://en.wikibooks.org/wiki/LaTeX/Algorithms}{Algorithms} 与 \pkg{algorithm2e} 文档,给出一个简单的示例,见算法 \ref{alg:alg1}。

\begin{algorithm}
  \SetAlgoLined
  \KwResult{Write here the result}
  initialization\;
  \While{While condition}{
    instructions\;
    \eIf{condition}{
      instructions1\;
    }{
      instructions3\;
    }
  }
  \caption{如何写算法}\label{alg:alg1}
\end{algorithm}

\section{绘图}
关于使用 \LaTeX{} 绘图的更多例子,请参考 \href{https://www.overleaf.com/learn/latex/Pgfplots_package}{Pgfplots package}。一般建议使用如 \app{Photoshop}、\app{PowerPoint} 等制图,再转换成 \fmt{PDF} 等格式插入。

\section{单位}
单位的输入请使用 \pkg{siunitx} 包中提供的 \verb|\si| 与 \verb|\SI| 命令。在以前,\LaTeX{} 中输入角度需要使用 \verb|$^\circ$| 的奇技淫巧,现在只需要 \verb|\ang| 命令解决问题。当然 \pkg{siunitx} 包中还提供了不少其他有用的命令,有需要的可以自行阅读 \pkg{siunitx} 文档。

\section{物理符号}
\WhuThesis 亦使用了 \pkg{physics} 宏包,旨在让用户更加方便、简洁地使用、输入物理符号。示例如下
\begin{equation}
  \begin{aligned}
    \int_0^\frac{\symup{\pi}}{2} \abs{\sin{x}} \dd{x} & = 2 \int_0^{\symup{\pi}} \sin{x} \dd{x} \\
                                                      & = -2 \eval{\cos{x}}_0^{\symup{\pi}}     \\
                                                      & = 4
  \end{aligned}
\end{equation}

\section{写在最后}
工具不重要,对工具的合理运用才重要。希望本模板对大家的论文写作有所帮助。
